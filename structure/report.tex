
\usepackage[T1]{fontenc}
\usepackage[brazil]{babel}
\usepackage[utf8]{inputenc}
\usepackage{color, ucs}
\usepackage[table, xcdraw, dvipsnames]{xcolor}
\usepackage{subfigure}
\usepackage{booktabs}
\usepackage{graphics, graphicx}
\usepackage[intlimits]{amsmath}
\usepackage{xfrac}
\usepackage{amssymb}
\usepackage{wrapfig}
\usepackage{paralist} 
\usepackage{pgffor}
\usepackage{authblk}
\usepackage[hang, small, labelfont=bf, labelsep=period, up, textfont=it, up]{caption} 
\usepackage{indentfirst}
\usepackage{abstract} 
\usepackage{titlesec} 
\usepackage{microtype} 
\usepackage{siunitx}
\usepackage[version=4]{mhchem}
\usepackage{pdfpages}
\usepackage{appendix}
\usepackage{blindtext}
\usepackage{bm}
\usepackage{enumitem}
\usepackage{mdframed}
\usepackage{environ}
\usepackage{mathtools}
\usepackage[pdftex]{hyperref}
\usepackage{lipsum} 
% --------------------
%     REFERENCIAS
%\usepackage{natbib}
\usepackage[alf, abnt-emphasize=bf]{abntex2cite}
\usepackage{quoting}
% --------------------
\usepackage{tikz}
\usepackage{url}
\usepackage{afterpage}
\usepackage{fancybox}
\usepackage{mathptmx}
%\usepackage{lmodern}
%\usepackage{times}
\usepackage{setspace}
\usepackage{multirow}

\usepackage{textcomp}
\usepackage{xparse}

\usepackage{comment}
\usepackage[left=3cm,top=3cm,right=2cm,bottom=2cm]{geometry}

\hypersetup{
    hidelinks
}

% Imagem \ Ajuste do tamanho da imagem
\def \imagem {logoufu.png}\def \ajuste {.11
}

% Cabeçalho:
\def \cabecalho {
universidade federal de uberlândia\\
instituto de física\\
gradução em engenharia eletrônica e de telecomunicações
}

% Nomes:
%% Exemplo:
% \def \nome {MARQUES, C. dos S.;\hfill OLIVEIRA, F. G.;\hfill CAMELO, L. M.;\hfill TEIXEIRA, K. R. S.
% }

% Para Fulano da silva
\def \nome {
Luis Miguel Alves Borges  (42011ETE011) \\ 
Mateus Barros Lorêdo (42011ETE022)\\
Matheus Felipe Lima (41921ETE006)\\
Vinicius Santos Ramos (41521ETE002)
}

% Titulo:
\def \titulo {
Relatório técnico de prática laboratorial 5:\\
DETERMINAÇÃO DO CALOR ESPECÍFICO DA ÁGUA
}



% Local e ano:
\def \local {Patos de Minas - MG}\def \ano {16/02/2022
}

% Introdução
\def \nota {O experimento prático se propõe a obter o calor específico da água dada certas circunstâncias.
}

% Professor
\def \prof {Prof. Dr. Ricardo Ribeiro de Avila
}

\onehalfspacing     % espaçamento de 1.5 entre linhas.
\setlength{\parindent}{1.3cm}

\addtocontents{toc}{~\hfill\textbf{Páginas}\par} % Sumário com o nome pagina ao final.

\newcommand{\uveci}{{\bm{\hat{\textnormal{\bfseries\i}}}}} % Utilize \uveci para o vetor unitário i
\newcommand{\uvecj}{{\bm{\hat{\textnormal{\bfseries\j}}}}} % Utilize \uvecj para o vetor unitário j
\newcommand{\uveck}{{\bm{\hat{\textnormal{\bfseries k}}}}} % Utilize \uveck para o vetor unitário k
\DeclareRobustCommand{\uvec}[1]{{%
  \ifcsname uvec#1\endcsname
     \csname uvec#1\endcsname
   \else
    \bm{\hat{\mathbf{#1}}}%
   \fi
}}
% pagina em branco com "\paginabranco""
\newcommand\paginabranco {\null\thispagestyle{empty}\addtocounter{page}{-1}\newpage}
 
% Um pequeno atalho para a imagem na capa 
\providecommand{\imagemlogo}[2]{\begin{figure}[!h]\centering\includegraphics[scale=#1]{figuras/#2}\end{figure}}

% A função \chamafig{local da sua figura aqui}{legenda aqui}{referencia aqui} chama a sua figura feita em TIZ.
\providecommand{\chamafig}[3]{\begin{figure}[!h]\centering\input{#1}\caption{#2}\label{#3}\end{figure}}

% A função \chamaimg{tamanho}{local da sua imagem aqui}{legenda aqui}{referencia aqui} chama a sua imagem.
\providecommand{\chamaimg}[4]{\begin{figure}[!h]\centering\includegraphics[width=#1\textwidth]{#2}\caption{#3}\label{#4}\end{figure}}

% Estrutura da capa
\def \capa {\thispagestyle{empty}\begin{center}\begin{figure}[h!]\centering\includegraphics[scale=\ajuste]{figure/\imagem}\end{figure}{\MakeUppercase\cabecalho\vspace{1.65cm}}\\{\MakeUppercase\nome}\vfill{\large\MakeUppercase\titulo}\vfill{\local\\\ano}\end{center}\newpage}

% Estrutura da contra capa
\def \contracapa {\thispagestyle{empty}\begin{center}{\MakeUppercase\nome}\vfill{\large\MakeUppercase\titulo\vspace{2.5cm}}\\{\raggedleft\begin{minipage}{.6\textwidth}\nota\\ \ \\\prof.\end{minipage}\par}\vfill{\local\\\ano}\end{center}\newpage}

% ajuste da imagem
\ExplSyntaxOn
    \prop_new:N \MainName
    
    \NewDocumentCommand{\definetypedocument}{mmmm}{
      \prop_gput:Nnn \MainName{#1-first}{#2}
      \prop_gput:Nnn \MainName{#1-second}{#3}
      \prop_gput:Nnn \MainName{#1-third}{#4}
    }
    \NewDocumentCommand{\definefirst}{m}{
      \prop_item:Nn \MainName{#1-first}
    }
    \NewDocumentCommand{\definesecond}{m}{
      \prop_item:Nn \MainName{#1-second}
    }
    \NewDocumentCommand{\definethird}{m}{
      \prop_item:Nn \MainName{#1-third}
    }
\ExplSyntaxOff

% ajuste da capa
\newcommand\acp {\vspace{\ajustecapa}}

% ajuste da folha de rosto
\newcommand\fdrt {\vspace{\ajustefolharosto}}

% Estrutura da Referencias
%\providecommand {\Areferencias}[1]{\nocite*\bibliographystyle{plainnat}\bibliography{#1}}

% Estrutura do sumário
\def \sumario {\thispagestyle{empty}\tableofcontents\newpage}
    
% Estrutura da lista de figuras
\def \listafiguras {\thispagestyle{empty}\listoffigures\newpage}

% Estrutura da lista de tabelas
\def \listatabelas {\thispagestyle{empty}\listoftables\newpage}

\def \documentconfig {\setlength{\parskip}{.5cm}}

%\setlength{\parindent}{1.25cm}
